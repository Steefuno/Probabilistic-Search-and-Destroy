%https://www.overleaf.com/learn/latex/Creating_a_document_in_LaTeX
%https://oeis.org/wiki/List_of_LaTeX_mathematical_symbols
\documentclass[12pt]{report}
\usepackage{graphicx}

\title{CS 440: Probabilistic Search}
\author{Steven Nguyen \& Kyra Kennedy}
\date{10 April 2021}

\begin{document}

\begin{titlepage}
\maketitle
\end{titlepage}

\section{Abstract}
In this project, we demonstrated searching for a cell given probabilities of its location and using prior beliefs while iterating.

\section{Academic Integrity}
For this project, Steven Nguyen handled problems 1 and 2, as well as, Agent 2. Kyra Kennedy worked on creating the environment, Agent 1, and the Improved Agent. Both contributed to the making of the report. \\
I, Steven Nguyen, have not copied our code or taken from online or another student's work. \\
I, Kyra Kennedy, have not copied our code or taken from online or another student's work.

\section{Problem 1}
Assume:\begin{itemize}
	\item Belief$[C_{i}]_{t}$ is the belief that the target is in cell $C_{i}$ at time {t}. It is calculated using $P(in(C_{i})|O_{t})$.
	\item $P(F(C_{j}))$ is the probability that the target is not found when searching cell $C_{j}$. 
		It is calculated using $P(F(C_{j})|in(C_{j}))*P(in(C_{j}))+P(!in(C_{j}))$, the probability that the target is 
		not found in $C_{j}$ if the target is in $C_{j}$ added to the probability that the target is not in $C_{j}$.
	\item $P(in(C_{i}))$ is the probability that the target is in cell $C_{i}$, $1/$the number of cells.
	\item $O_{t}$ is Observations at time $t$.
	\item $N_{i,t}$ is the number of times cell $C_{i}$ has been observed and resulted in a failure at time $t$.
	\item Belief$[C_{i}]_{t+1}$, is belief that the target is in cell $C_{i}$ after applying a new observation, that the target is not at $C_{j}$. It is calculated by $P(in(C_{i})|O_{t}\land F(C_{j}))$.
\end{itemize}
First, for the cell that was just observed as a failure, Belief$[C_{j}]_{t}$ will change to Belief$[C_{j}]_{t+1}$, or $P(in(C_{i})|O_{t+1})$.\\
\\
This can be converted to $P(O_{t+1}|in(C_{j}))*P(in(C_{j}))/P(O_{t+1})$.\\
\\
$P(O_{t+1}|in(C_{j}))$ converts to $P(F(C_{j})|in(C_{j}))^{N_{j}}$ because given that the target is in $C_{j}$, the probabilities of not finding the target in all other cells becomes 1. Then, the probability of not finding the target in $C_{j}$ is $P(F(C_{j})|in(C_{j}))^{N_{j}}$ because the cell is searched $N_{j}$ times and each had $P(F(C_{j})|in(C_{j}))$ to fail.\\
\\
$P(O_{t+1})$ converts to $P(O_{t})*(P(in(C_{j}))*P(F(C_{j})|in(C_{j}))+(1-P(C_{j})))$.\\
$(P(in(C_{j}))*P(F(C_{j})|in(C_{j}))+(1-P(C_{j})))$ is the probability of either failing a search while the target is in $C_{j}$ or of the target not being in $C_{j}$. $P(O_{t})$ is multiplied by this probability to find the overall probability of the previous events AND the new observed event happening.\\
\\
Dividing Belief$[C_{j}]_{t+1}$ by Belief$[C_{j}]_{t}$ will give the multiplicative difference between Belief$[C_{j}]_{t}$ and Belief$[C_{j}]_{t+1}$.\\
This is: $P(F(C_{j})|in(C_{j}))^{1}/(P(in(C_{j}))*P(F(C_{j})|in(C_{j}))+(1-P(C_{j})))$\\
\\
So, multiplying the current belief by $P(F(C_{j})|in(C_{j}))^{1}/(P(in(C_{j}))*P(F(C_{j})|in(C_{j}))+(1-P(C_{j})))$ will update the belief for cell $C_{j}$.\\
\\
This changes the total belief on the map by $\Delta$Belief$[C_{j}]$, so dividing the absolute value of $\Delta$Belief$[C_{j}]$ by the number of cells excluding $C_{j}$ will be the belief all other cells will each increase by.

\section{Problem 2}
The probability that the target will be found in cell $C_{i}$ given observations is:\\
Belief$[C_{i}]*(1-P($Target not found in Cell$_{i}|$Target is in Cell$_{i}))$

\section{Problem 3}
Agent 1 had an average score of \\
Agent 2 had an average score of 115147.

\section{Problem 4}
For the improved agent, the probabilities were based on Agent 2, so querying cells that had the highest probability of finding the target. To improved upon this, if a cell was queried and return False (so the target was not found), it would be queried at least one more time. After being queried twice, if the cell still returned False, the agent checks the terrain and makes a decision from there. For the forest and maze of cave terrains, the cell is queried again, and if it fails again the cell is queried one last time, for a total of four. \\
This was done because the forest and mave of cave terrains are more likely to return false negatives, so they are checked the hardest. Once the cell has been queried two or four times (and returns False each time), the probabilities are updated, and the agent searches again. \\
Given all the resources, to make the agent even better, when a cell is going to be searched, it would be queried enough times to make it statistically improbable that the target is contained in that cell. By doing so, cells could be completely removed from the realm of possibility so the agent could essentially shrink the environment, giving it less cells to consider.

\end{document}